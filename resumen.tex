\documentclass[10pt,a4paper]{article}
\usepackage{blindtext}
\usepackage{subcaption}
\usepackage{graphicx}
\usepackage{tikz}
\usepackage{amssymb}
\usepackage{caption}
\usepackage{amsmath}
\usepackage{circuitikz}
\usepackage{hyperref}
\usepackage{amssymb}
\usepackage{amsmath}
\usepackage{listings}

\lstset{
    inputencoding=utf8,
    extendedchars=true,
    literate={á}{{\'a}}1 {é}{{\'e}}1 {í}{{\'i}}1 {ó}{{\'o}}1 {ú}{{\'u}}1 {ñ}{{\~n}}1 {Á}{{\'A}}1 {É}{{\'E}}1 {Í}{{\'I}}1 {Ó}{{\'O}}1 {Ú}{{\'U}}1 {Ñ}{{\~N}}1
}
\input{AEDmacros}
\newcommand{\notimplies}{\;\not\!\!\!\implies}
\title{Álgebra I}
\author{Tomás Agustín Hernández}
\date{}

\begin{document}
\maketitle

\begin{figure}[b]
    \centering
    \begin{tikzpicture}[remember picture,overlay]
        \node[anchor=south east, inner sep=0pt, xshift=-1cm, yshift=2cm] at (current page.south east) {
            \begin{minipage}[b]{0.5\textwidth}
                \includegraphics[width=\linewidth]{logo_uba.jpg}
                \label{fig:bottom}
            \end{minipage}
        };
    \end{tikzpicture}
\end{figure}

\newpage
\section*{Conjuntos}
Los conjuntos almacenan elementos, \textbf{no se consideran repetidos ni tampoco importa el orden}. Responde a la pregunta de $"$¿está el elemento?$"$, esto último quiere decir que no tenemos forma de tomar un elemento sino predicar acerca de si está o no. \\
\textbf{Ej.}: $ A = \{1, 2\}, B = \{2, 1\}$. A y B son considerados iguales, pues no importa el orden sino los elementos que tienen dentro.
\subsection*{Pertenecencia a un Conjunto}
Si consideramos cualquier elemento $x$, decimos que está en un conjunto A si \textbf{x pertenece a A}. \\
La pertenencia de un elemento a un conjunto la denotamos como: $x \in A$ \\
\textbf{Importante}: La relación está dada por $Elemento \ y \ Conjunto$ \\
Véase \hyperref[subsec:pertenecencia_conjuntos]{\underline{ánexo}} para ejemplos más didácticos.
\subsection*{Inclusión a un Conjunto}
Sean A y D conjuntos cualesquiera.
Decimos que D es un subconjunto de A sí y solo sí todos los elementos de D están en A. \\
La inclusión en un conjunto la denotamos como $D \subseteq A$ \\
Es posible leer el símbolo $ \subseteq $ de tres maneras: 
\begin{itemize}
    \item "D es un subconjunto de A"
    \item "D está incluido en A"
    \item "D está contenido en A"
\end{itemize}
Los subconjuntos posibles no salen más que haciendo combinaciones con sus elementos, es decir, agruparlos de diferentes formas. \\
Véase \hyperref[subsec:inclusion_conjuntos]{\underline{ánexo}} para ejemplos más didácticos.
\subsection*{Cardinal de un Conjunto}
Sea A un conjunto, el cardinal de un conjunto indica la cantidad de elementos en el conjunto. \\
Se denota como: $\#A$
\subsection*{Conjunto de Partes}
Se llama A un conjunto, se llama conjunto de partes al conjunto con todos los subconjuntos de A. \\
Se denota como $P(A)$
\begin{itemize}
    \item El elemento vacío $\emptyset \in P(A)$
    \item $A \in P(A)$
    \item $\#P(A) = 2^{\#A}$
\end{itemize}
\subsection*{Cantidad de Subconjuntos posibles dado un Conjunto}
Sea un conjunto A, la cantidad de subconjuntos D para el conjunto A es: $2^{\#A}$
\subsection*{Elemento Vacío}
Se representa con el símbolo de $\emptyset$. El elemento vacío está \textbf{incluido} en todos los conjuntos. \\
\textbf{Importante}: El elemento vacío NO pertenece a todos los conjuntos sino que está incluido en todos.
\subsection*{Cuantificadores}
Nos permiten predicar acerca de los elementos de un conjunto dado. 
\begin{itemize}
    \item $\forall$ x: Para todo x.
    \begin{itemize}
        \item Para que sea verdadero todos deben cumplir la condición dada.
        \item Es falso si existe un caso en que no se cumple.
    \end{itemize}
    \item $\exists$ x: Existe un x
    \begin{itemize}
        \item Para que sea verdadero alcanza con encontrar un caso verdadero.
        \item Es falso si no hay ningun caso que cumpla la condición
    \end{itemize}
\end{itemize}
\textbf{Importante}: El símbolo de $:$ o $/$ significa "tal que" \\
Véase \hyperref[subsec:cuantificadores]{\underline{ánexo}} para ejemplos más didácticos.
\section*{Operaciones entre Conjuntos}
Sean A y B conjuntos cualesquiera. La cantidad de filas que tendrá una tabla de verdad es: \textbf{$2^{cantVariables}$} \\
\textbf{Importante}: Las operaciones entre conjuntos que vamos a ver están relacionadas con la lógica proposicional.
\subsection*{Unión $(A \cup B)$}
Es exactamente igual como en la lógica proposicional. La unión es un $o$ lógico. En el conjunto resultante quedan los elementos de A y B. \\

\begin{table}[h!]
    \centering
    \begin{tabular}{|c | c | c|}
    \hline
    \textbf{A} & \textbf{B} & \textbf{$A \cup B$} \\[0.1cm]
    \hline
    V & V & V \\
    V & F & V \\
    F & V & V \\
    F & F & F \\
    \hline
    \end{tabular}
    \caption{Unión de conjuntos}
\end{table} 
Cada fila se puede generalizar para un x cualquiera en las operaciones lógicas. \\
\textbf{Ej.}: Si $x \in A \land x \in B$ entonces $ x \in A \cup B$ esto claramente nos dice que estamos en el caso de la fila 1. \\
\textbf{Ej.}: Si $x \notin A \land x \in B$ entonces $ x \in A \cup B$ esto claramente nos dice que estamos en el caso de la fila 3.
\subsection*{Intersección $(A \cap B)$}
Es exactamente igual como en la lógica proposicional. La intersección es un $"y"$ lógico. En el conjunto resultante quedan los elementos que están tanto en A y en B.
\begin{table}[h!]
    \centering
    \begin{tabular}{|c | c | c|}
    \hline
    \textbf{A} & \textbf{B} & \textbf{$A \cap B$} \\[0.1cm]
    \hline
    V & V & V \\
    V & F & F \\
    F & V & F \\
    F & F & F \\
    \hline
    \end{tabular}
    \caption{Intersección de conjuntos}
\end{table} \\
Cada fila se puede generalizar para un x cualquiera en las operaciones lógicas. \\
\textbf{Ej.}: Si $x \in A \land x \in B$ entonces $ x \in A \cap B$ esto claramente nos dice que estamos en el caso de la fila 1. \\
\textbf{Ej.}: Si $x \notin A \land x \in B$ entonces $ x \notin A \cap B$ esto claramente nos dice que estamos en el caso de la fila 3. \\
\textbf{Importante}: Si la intersección entre dos conjuntos es vacía $(\emptyset)$ se dice que son conjuntos disjuntos.
\subsection*{Complemento $(A \cap B)$}
En la lógica proposicional, el complemento es la negación. Lo que está en un conjunto universal V pero no en el conjunto.
\begin{table}[h!]
    \centering
    \begin{tabular}{|c | c|}
    \hline
    \textbf{A} & \textbf{$\neg A$} \\[0.1cm]
    \hline
    V & F \\
    V & F  \\
    F & V  \\
    F & V \\
    \hline
    \end{tabular}
    \caption{Complemento en Conjuntos}
\end{table} \\
Cada fila se puede generalizar para un x cualquiera en las operaciones lógicas. \\
\textbf{Ej.}: Si $x \in A$ entonces termina siendo $ x \notin A$ esto claramente nos dice que estamos en el caso de la fila 1. \\

Sea $A=\{1, 2\}, B=\{3, 4, 5\}, C=\{8, 9\}, V=\{A, B, C\} \implies A^{c} = \{3, 4, 5, 8, 9\}$ \\
\textbf{Importante}: Nótese que siempre se hace el complemento en base a los elementos que hay en el universo y se excluyen algunos. En este caso, del universo V nos quedamos con los que NO están en A.
\newpage
\subsection*{Diferencia $(A-B )$}
Esta operación es conocida también de la siguiente manera $A\symbol{92}B$.
Es una equivalencia de $A \cap B^{c}$. Representa lo que está en A pero no en B. Si se lo quisiera representar en la tabla de verdad, debe representar la equivalencia.
\begin{table}[h!]
    \centering
    \begin{tabular}{|c | c | c | c|}
    \hline
    \textbf{A} & \textbf{B} & \textbf{$B^{c}$} & \textbf{$A \cap B^{c}$} \\[0.1cm]
    \hline
    V & V & F & F\\
    V & F & V & V\\
    F & V & F & F\\
    F & F & V & F\\
    \hline
    \end{tabular}
    \caption{Diferencia de conjuntos}
\end{table} 
\subsection*{Diferencia Simétrica $(A \triangle B)$}
Equivalente al XOR($\veebar$) u $o$ excluyente en la lógica proposicional. \\
Es una equivalencia de $(A - B) \cup (B - A)$ y $(A\cup B) - (A \cap B)$. Representa lo que está en A o en B pero no en ambos. 
\begin{table}[h!]
    \centering
    \begin{tabular}{|c | c | c | c | c|}
    \hline
    \textbf{A} & \textbf{B} & \textbf{$A \veebar B$} & \textcolor{blue}{$(A - B) \cup (B - A)$} & \textcolor{blue}{$(A\cup B) - (A \cap B)$} \\[0.1cm]
    \hline
    V & V & F & F & F \\
    V & F & V & V & V \\
    F & V & V & V & V \\
    F & F & F & V & V \\
    \hline
    \end{tabular}
    \caption{Diferencia Simétrica en conjuntos}
\end{table} \\
\textbf{Nota}: Las columnas en azul son equivalencias a la operación $\veebar$ y son útiles a la hora de demostrar. \\
\textbf{Ej.}: Si $x \in A \land x \in B$ entonces $ A \veebar B = F$ esto claramente nos dice que estamos en el caso de la fila 1.
\textbf{Ej.}: Si $x \in A \land x \notin B$ entonces $ A \veebar B = V$ esto claramente nos dice que estamos en el caso de la fila 2. \\
\subsection*{Inclusión $(A \subseteq B)$}
Representa el $\implies$ de la lógica proposicional. Recordemos que la inclusión es verdadera si todos los elementos de A están en B siendo A y B conjuntos cualesquiera. \\
Es lo que vamos a utilizar para demostrar, y es importante que se lo entienda bien. 
\begin{table}[h!]
    \centering
    \begin{tabular}{|c | c | c|}
    \hline
    \textbf{A} & \textbf{B} & \textbf{$A \implies B$} \\[0.1cm]
    \hline
    V & V & V \\
    V & F & F \\
    F & V & V \\
    F & F & V \\
    \hline
    \end{tabular}
    \caption{Inclusión de conjuntos}
\end{table} 
\begin{itemize}
    \item El único caso que nos importa es que si el antecedente es verdadero, hay que ver que el consecuente NO sea falso. En las demostraciones asumimos que vale el antecedente y tenemos que ver si hace verdadero al consecuente.
    \item Si no se cumple el antecedente, el consecuente es siempre verdadero.
\end{itemize}
Cada fila se puede generalizar para un x cualquiera en las operaciones lógicas. \\
\textbf{Ej.}: Sea $A = \{1, 2, 3\} \ B = \{10, 40\} \ x = 100$ ¿Se cumple que $ x \in A \implies x \in B$? ¿100 está en A? No, y al ser una implicación si el antecedente no se cumple, queda toda la proposición verdadera. Luego, sí, se cumple que $ x \in A \implies x \in B$. Esto claramente nos dice que estamos en el caso de la fila 3. \\ 
\textbf{Ej.}: Sea $A = \{1, 2, 3\} \ B = \{10, 40\} \ x = 3$ ¿Se cumple que $ x \in A \implies x \in B$? ¿3 está en A? Sí. Entonces esto hace al antecedente verdadero ¿me basta para decir que la proposición es verdadera? No. Primero debo ver qué pasa con el consecuente. ¿Es cierto que 3 está en B? No. Entonces como el antecedente es verdadero y el consecuente es falso, la proposición es falsa. Luego, no, no se cumple que $ x \in A \implies x \in B$. Esto claramente nos dice que estamos en el caso de la fila 2. \\
\textbf{Nota}: Que se entiendan los ejemplos anteriormente mencionados es realmente importante. Se usa en prácticamente todas las demostraciones.\subsection*{Igualdad $(A \iff B)$}
Representa el $\iff$ (sí y solo sí) de la lógica proposicional. Recordemos que la igualdad es verdadera si todos los elementos de A están en B siendo A y B conjuntos cualesquiera. \\
\begin{table}[h!]
    \centering
    \begin{tabular}{|c | c | c|}
    \hline
    \textbf{A} & \textbf{B} & \textbf{$A \iff B$} \\[0.1cm]
    \hline
    V & V & V \\
    V & F & F \\
    F & V & F \\
    F & F & F \\
    \hline
    \end{tabular}
    \caption{Igualdad de conjuntos}
\end{table} 
\begin{itemize}
    \item La manera de demostrar esto es viendo si se cumple que $A \subseteq B \ y \ B \subseteq A$
\end{itemize}
Cada fila se puede generalizar para un x cualquiera en las operaciones lógicas. 
\subsection*{Leyes de De Morgan}
La forma más fácil de verlo es que se distribuye el complemento y se invierte la operación.
\begin{itemize}
    \item $(A \cup B)^{c} = A^{c} \cap B^{c}$
    \item $(A \cap B)^{c} = A^{c} \cup B^{c}$
\end{itemize}
\subsection*{Propiedades de Conjuntos}
\begin{itemize}
    \item Distributiva: $A \cap (B \cup C) = (A \cap B) \cup (A \cap C)$
    \item Conmutatividad: $A \cap B = B \cap A $ Igual para unión
    \item Conjuntos Disjuntos $ A \cap B = \emptyset$
\end{itemize}
TODO: Agregar en Anexo demostración de distributividad.
\subsection*{Producto Cartesiano ($A X B$)}
Sean dos conjuntos A y B cualquiera. El producto cartesiano es el par ordenado (c, d) con $c \in A$ y $d \in B$. \\ 
La cantidad de elementos máxima en un producto cartesiano es = $\#A \ast \#B$. \\
Sí o sí es necesario que el par NO sea nulo, es decir, deben ser elementos válidos. \\
\textbf{Importante}: AXB $\neq$ BXA \\
\textbf{Ej.}: $A = \{1, 2, 3\}, B = \emptyset, AXB = \emptyset$, pues B está vacío. \\
\textbf{Ej.}: $A = \{1, 2, 3\}, B = \{4, 5\}, \#AXB = 6, AXB = \{\{1, 4\}, \{1, 5\}, \{2, 4\}, \{2, 5\}, \{3, 4\}, \{3, 5\} \}$
\section*{Relaciones}
Sean A y B conjuntos. Una relación A en B es un subconjunto cualquiera R de AXB. \\
\textbf{Ej.}: $A = \{1\}, B = \{4, 5\}, AXB=\{\{1, 4\}, \{1, 5\}\}, R = \{(1, 1), (1, 2), (1, 4)\}$ ¿Es R una relación válida de AXB? No, no lo es pues $(1, 1) \in R$ pero $ (1, 1) \notin AXB$
\section*{Relaciones de un conjunto en sí mismo}
Sea A un conjunto cualquiera. Se dice que A está relacionado con A sí y solo sí AXA. \\
Se dice que R es una relación en A cuando $ R \subseteq AXA$ \\
\textbf{Ej.}: $A = \{1, 2, 3\}, AXA=\{\{1, 1\}, \{1, 2\}, \{1, 3\}, \{2, 2\}, \{2, 3\}, \{3, 3\}\}, R = \{\{1, 2\}, \{1, 4\}\}$ ¿Es R una relación válida de AXB? No, no lo es pues $\{1, 4\} \in R$ pero $ \{1, 4\} \notin AXB$ \\
\textbf{Ej.}: $A = \{1, 2, 3\}, AXA=\{\{1, 1\}, \{1, 2\}, \{1, 3\}, \{2, 2\}, \{2, 3\}, \{3, 3\}\}, R = \{(1, 1), (1, 2), (1, 3)\}$ ¿Es R una relación válida de AXB? Sí lo es, pues todos los subconjuntos pertenecientes a R pertenecen a AXA. \\
Veamos ahora \textbf{las propiedades de las relaciones de un conjunto en sí mismo}.
\subsection*{Reflexividad}
Una relación es reflexiva sí y solo sí para todo elemento de A, a está relacionado con A. \\
\textbf{Formalmente}: $ \forall a \in A \implies aRa$ \\
\textbf{Ej.}: $A = \{1, 2, 3\}, AXA=\{\{1, 1\}, \{1, 2\}, \{1, 3\}, \{2, 2\}, \{2, 3\}, \{3, 3\}\}, R = \{(1, 1), (1, 2), (1, 3)\}$ ¿Es R una relación válida de AXB? Sí lo es. ¿Es reflexiva? No, no lo es, pues 2 no está relacionado con 2, ni tampoco 3 con el 3.\\
\textbf{Ej.}: $A = \{1, 2, 3\}, AXA=\{\{1, 1\}, \{1, 2\}, \{1, 3\}, \{2, 2\}, \{2, 3\}, \{3, 3\}\}, R = \{(1, 1), (2, 2), (3, 3), (1, 2)\}$ ¿Es R una relación válida de AXB? Sí lo es. ¿Es reflexiva? Sí, pues para todo elemento a en R, aRa.\\

\textbf{Nota}: Una relación que solamente tiene dentro los elementos aRa es llamada identidad. Considerando el AXA anterior, la relación identidad sería R = $\{(1, 1), (2, 2), (3, 3)\}$  \\
\textbf{Nota}: Si se quisiera buscar un contraejemplo, una buena forma es hallar un elemento que no se relacione con si mismo. 
\subsection*{Simetría}
Sean a, b $\in$ A. Una relación es simétrica sí y solo sí $aRb \implies bRa$. Vulgarmente decimos que si uno está relacionado con el otro, el otro está obligado a estarlo también. \\
\textbf{Formalmente}: $ \forall a, b \in A \ \symbol{92} \ aRb \implies bRa$ \\
\textbf{Ej.}: $R = \{(1, 2), (3, 1)\}$, no es simétrica pues sucede que $1R2$ pero 2 no está relacionado con 1. \\
\textbf{Ej.}: $R = \{(1, 2), (2, 1)\}$, es simétrica pues para todo elemento relacionado, se relacionan conjuntamente. \\
\textbf{Ej.}: $R = \{(1, 1)\}$, es simétrica, pues no existe ninguna relación entre elementos diferentes. Por lo tanto, el antecedente es falso, luego la proposición ($aRb \implies bRa$) es verdadera \\

\textbf{Nota}: Como es una implicación, si el antecedente es falso (no hay ningún elemento, o no existe relación entre ellos) entonces es simétrica.  \\
\textbf{Nota}: Si se quisiera buscar un contraejemplo, una buena forma es buscar simplemente un elemento que se conecte con otro, pero no al revés.
\subsection*{Antisimétrica}
Sean a, b $\in$ A. Una relación es antisimétrica sí y solo sí $aRb \land bRa \implies A = B$. Vulgarmente decimos que si ambos están relacionados, entonces es porque son iguales. \\
\textbf{Formalmente}: $ \forall a, b \in A \ \symbol{92} \ aRb \land bRa \implies a=b$ \\
\textbf{Ej.}: $R = \{(1, 2), (2, 1)\}$, no es antisimétrica pues 1 se relaciona con 2, y 2 se relaciona con 1 pero $1 \neq 2$ \\
\textbf{Ej.}: $R = \{(1, 1), (2, 2)\}$, es antisimétrica pues 1 se relaciona con 1, 2 se relaciona con dos y son los mismos elementos. \\

\textbf{Nota}: Si se quisiera buscar un contraejemplo, una buena forma es buscar un conjunto que la haga simétrica considerando la relación entre elementos diferentes.
\subsection*{Transitividad}
Sean a, b $\in$ A. Una relación es transitiva sí y solo sí $aRb \land bRc \implies aRc$. Vulgarmente decimos que si a me conecta con la calle b, y b con la calle c, entonces a me lleva a c. \\
\textbf{Formalmente}: $ \forall a, b \in A \ \symbol{92} \ aRb \land bRc \implies aRc$ \\
\textbf{Ej.}: $R = \{(1, 2), (2, 3), (1, 3) \}$, es transitiva pues como 1 me conecta con 2, y 2 se conecta con 3, entonces desde 1 puedo llegar a 3. \\
\textbf{Nota}: Si se quisiera buscar un contraejemplo, una buena forma es buscar un a que esté relacionado con un b, y ese b esté relacionado con un c pero a no esté relacionado con c. Básicamente, sería hacer que se cumpla el antecedente pero no el consecuente.
\subsection*{Relación Identidad}
Dado un conjunto A relacionado y en sí mismo AXA. Una relación R es identidad sí y solo sí todos los elementos de R cumplen la forma de (a, a). \\
\textbf{Ej.}: $R = \{(1, 1), (2, 2), (3, 3)\}$. Es identidad. \\
\textbf{Ej.}: $R = \{(1, 2), (3, 3) \}$. No es identidad pues $1 \neq 2$.
\subsection*{Relación Total}
Dado un conjunto A. Una relación R es total cuando R = AXA.
\subsection*{Clases de Equivalencia}
\label{subsec:clases_equivalencia}
Sea A un conjunto y una relación de equivalencia en A. \\
Para cada $x \in A$, la clase de equivalencia de x es el conjunto: $\bar{x} = [x] = \{y \in A \ \symbol{92} \ yRx\} \subseteq A$. Vulgarmente hablando, es el conjunto de los elementos con los que se relaciona x.
\textbf{Ej.}: $ R = \{(1,1), (1, 2), (2, 1), (3, 1), (1, 3), (4, 5), (5, 4), (6, 6)\}$
\begin{itemize}
    \item $[1] = [2] = [3] =  \{1, 2, 3\}$
    \item $[4] = [5] = \{4, 5\}$
    \item $[6] = \{6\}$
\end{itemize}
\textbf{Propiedad fundamental}: O Sucede $ \bar{x} = \bar{y}$ o sucede $\bar{x} \cap \bar{y} = \emptyset $ pero no ambas a la vez. \\
TODO: Añadir los ejemplos que hice en la tablet al anexo.
\subsection*{Representante de clase}
Cualquier elemento de una clase de equivalencia representa a su clase. 
En la sección de \hyperref[subsec:clases_equivalencia]{\underline{\textbf{Clases de Equivalencia}}} cada clase $[x]$ es representada por x. \\
\section*{Funciones}
Una función $A \leftarrow B$ es una relación $ f \subseteq AXB$ entre A y B que satisface: 
\begin{itemize}
    \item $\forall a \in A, \exists! \ b \in B \ / \ (a, b) \in f$. Si a la función le mando un a, me devuelve siempre el mismo b (no existe más de un b para un mismo a).
\end{itemize}
En las funciones, el valor de b está determinado por a, es decir, $b = f(a)$. \\
Para ser función, se debe cumplir \textbf{existencia ($\forall a \in A$)} y \textbf{unicidad ($\exists! \ b $)}. \\
\textbf{Dominio}: Son los valores los cuales podemos observar y enviar para que el codominio nos arroje un resultado. \\
\textbf{Codominio}: Son el resultado para un x dado que proviene del Dominio. \\
Veamos ahora \textbf{los distintos tipos de funciones}.
\subsection*{Función Inyectiva}
Sean x, y x'. Si sucede que f(x) = f(x') entonces x = x'. \\
Vulgarmente hablando, no existen dos x diferentes tal que el valor que arroja la imagen es el mismo.
\[\begin{minipage}[b]{0.9\textwidth}
    \includegraphics[width=\linewidth]{assets/def_inyectiva.jpg}
    \label{fig:def_inyectiva}
\end{minipage}\]
\subsection*{Función Sobreyectiva}
Sea $f:A \rightarrow B$. Una función es sobreyectiva si $ \forall b \in B, \exists a \ / \ f(a) = b$. \\
Vulgarmente hablando, todos los posibles valores del codominio corresponden con algún valor del dominio.
\[\begin{minipage}[b]{0.9\textwidth}
    \includegraphics[width=\linewidth]{assets/def_sobreyectiva.jpg}
    \label{fig:def_sobreyectiva}
\end{minipage}\]
\subsection*{Función Biyectiva}
Si es Inyectiva y Sobreyectiva a la vez, entonces es biyectiva.
\[\begin{minipage}[b]{0.9\textwidth}
    \includegraphics[width=\linewidth]{assets/def_biyectiva.jpg}
    \label{fig:def_biyectiva}
\end{minipage}\]
TODO: Añadir los ejemplos que hice en la tablet al anexo.
\subsection*{Composición de Funciones}
Sean las relaciones $R = AX\bar{B}$ y $S = \bar{B}XA$. \\
La composición de R y S es la relación de A a C dada por: 
$ SoR = \{(a, c) \in AXC, \ \exists b / (a, b) \in R \land (b, c) \in S\}$
\[\begin{minipage}[b]{0.9\textwidth}
    \includegraphics[width=\linewidth]{assets/composicion_funciones.png}
    \label{fig:composicion_funciones}
\end{minipage}\]
\textbf{Ej.}: $f:A\rightarrow B \ g:B\rightarrow C \ gof: A\rightarrow C$ \\
\textbf{Nota}: gof = g es el codominio y f el dominio. \\
Propiedades: 
\begin{itemize}
    \item Si $f:A\rightarrow B \land g:B\rightarrow C$ son funciones entonces $gof$ también.
    \item $I_{A}: A \rightarrow A$
    \item $I_{A} \ o \ f = f$ 
\end{itemize}
\subsection*{Función Inversible}
Si $f:A\rightarrow B$ su inversa es $g:B\rightarrow A$ y la denotamos como $f^{-1}$. \\
Decimos que una función f es inversible si es biyectiva y hay una función g tal que:
\begin{itemize}
    \item $fog = I_{B}$
    \item $gof = I_{A}$
\end{itemize}
Y en ese caso decimos que g es inversa de f. \\
\section*{Naturales - Sumatoria - Productoria}
\subsection*{Números Naturales}
Son un conjunto infinito. \\
Algunas propiedades:
\begin{itemize}
    \item Conmutatividad: $ m+n = n+m$
    \item Asociatividad: $(m+n)+k = m+(n+k)$ y $ (m\ast n)\ast k = m\ast (n\ast k)$
    \item Distributividad: $m(n+k) = mn + mk$
\end{itemize}
\subsection*{Sumatoria}
Permite indicar claramente cuantas veces hay que sumar algo dado: $ a_{1} + a_{2} + a_{3} + ... + a_{n-1} + a_{n} \equiv \sum_{i=1}^{n}{a_{i}}$ \\
Los límites superior e inferior de la sumatoria son \textbf{inclusivos} por lo tanto, el último caso es $i=n$.
Algunos casos bastante comunes:
\begin{itemize}
    \item Sumar 1 n veces: \[\sum_{i=0}^{n}{1} = n\]
    \item Sumar los n términos: \[\sum_{i=0}^{n}{i} = \frac{n(n+1)}{2}\]
\end{itemize}
\textbf{Importante}: Es posible manejar los términos de la sumatoria manipulando los límites. \\
Ej.: \[\sum_{i=\textbf{0}}^{\textbf{n+1}}{i} \equiv \sum_{\textbf{i=1}}^{\textbf{n}}{i} \] 
Nótese que en ambos se hace la misma cantidad de operaciones aunque hayamos quitado el n+1 y empezar la sumatoria desde 1 e ir hasta n, pero en el caso de la sumatoria de la derecha podemos aplicar un algoritmo conocido. \\
\textbf{Propiedades de la Sumatoria}: 
\begin{itemize}
    \item \textbf{Juntar sumatorias}: $\sum_{k=1}^{n}{a_{k}} + \sum_{k=1}^{n}{b_{k}}  \equiv \sum_{k=1}^{n}{a_{k} + b_{k}} $, como ambos van desde k = 1 hasta un n dado y los términos van en sincronía con k podemos juntarlos en una sola sumatoria.
    \item \textbf{Distribución denominador en suma}: $\sum_{k=1}^{n}{\frac{a+b+c}{d+c}} \equiv \sum_{k=1}^{n}{\frac{a}{d+c} + \frac{b}{d+c} + \frac{c}{d+c}}$
    \item \textbf{Quitar constante}: $\sum_{k=1}^{n}{c \ast a_{k}} \equiv c \ast \sum_{k=1}^{n}{a_{k}}$
    \item \textbf{Extraer términos}: $\sum_{k=1}^{2^{n+1}}{k^{2}} \equiv \sum_{k=1}^{2^{n} \ast 2}{k^{2}} \equiv \sum_{k=1}^{2^{n}}{k^{2}} + \sum_{k=2^{n}+1}^{2^{n+1}}{k^{2}} $, útil en inducción, lo veremos más adelante.
    \item \textbf{Agregar término}: $\sum_{i=k+1}^{n}{2i+1} \equiv (\sum_{i=k}^{n}{2i+1}) - 2(k+1)+1$, útil en inducción, lo que hacemos es que al agregar un término, se lo restamos en la suma final pues solo agregamos ese término por comodidad pero no se nos pedía en la suma original.
\end{itemize}
\subsection*{Suma de Gauss}
Un poco de contexto: ¿como hacemos para sumar los n números naturales? utilizamos la suma de gauss. \\
Es súper útil conocer esto porque he visto lugares donde suman los primeros n elementos utilizando un for. \\
\[\sum_{i=1}^{n}{i} = \frac{n(n+1)}{2} \] 
\textbf{Importante}: en la materia se utiliza todo el tiempo, por lo tanto no se recomienda memorizarla pero sí entenderla.
\subsection*{Serie Geométrica}
Un poco de contexto: ¿cómo hacemos para sumar los n términos que solo cambia el valor de la potencia? utilizamos la serie geométrica. \\
La serie geométrica se separa en dos casos, cuando $q=1$ y cuando $q\neq 1$
\[Q = \sum_{i=0}^{n}{q^{i}} = \left\{ \begin{array}{lcc} \frac{q^{n+1}-1}{q-1} & si & q \neq 1 \\ \\ n+1 & si & q = 1 \end{array} \right.\]
\subsection*{Productoria}
Mismo que la sumatoria pero de multiplicación: $ a_{1} \ast a_{2} \ast a_{3} \ast ... \ast a_{n-1} \ast a_{n} \equiv \prod_{i=1}^{n}{a_{i}}$ \\
\textbf{Importante}: en la materia se utiliza todo el tiempo, por lo tanto no se recomienda memorizarla pero sí entenderla. \\
Algunos casos bastante comunes:
\begin{itemize}
    \item Multiplicar los n términos: \[\prod_{i=1}^{n}{i} = n!\]
    \item Exponenciación: \[\prod_{i=1}^{n}{c} = c^{n}\]
\end{itemize}
\textbf{Importante}: Al igual que en la sumatoria, podemos manipular los límites superior e inferior. \\
\textbf{Propiedades de la Productoria}: 
\begin{itemize}
    \item \textbf{Juntar productorias}: $\prod_{k=1}^{n}{a_{k}} + \prod_{k=1}^{n}{b_{k}}  \equiv \prod_{k=1}^{n}{a_{k} + b_{k}} $, como ambos van desde k = 1 hasta un n dado y los términos van en sincronía con k podemos juntarlos en una sola productoria.
    \item \textbf{Quitar constante}: $\prod_{k=1}^{n}{c \ast a_{k}} \equiv c^{n} \ast \prod_{k=1}^{n}{a_{k}}$
    \item \textbf{Factor común}: $\prod_{k=1}^{n}{\alpha \ast a_{i} \ast b_{i}} \equiv  \alpha^{n} \ast \prod_{k=1}^{n}{a_{i} \ast b_{i}} \equiv \alpha^{n} \ast \prod_{k=1}^{n}{a_{i}} \ast \prod_{k=1}^{n}{b_{i}} $
    \item \textbf{Agregar término}: $\prod_{i=k+1}^{n}{2i+1} \equiv (\prod_{i=k}^{n}{2i+1}) \ast \frac{1}{2(k+1)+1}$, útil en inducción, lo que hacemos es que al agregar un término. Como acá es una productoria lo que hacemos es "dividir" en vez de restar, que a su vez dividir es multiplicar por 1/algo.
\end{itemize}
\section*{Inducción}
Nos permite poder probar cosas utilizando el álgebra. Es muy útil cuando necesitamos predicar que a partir de cierto valor una proposición se cumple siempre. Un conjunto inductivo es solamente $\nat$ \\
\textbf{Principio de Inducción}: Un subconjunto S de $\nat$ es inductivo si 
\begin{itemize}
    \item $1 \in S$
    \item $Si \ k \in S \implies k+1 \in S$
\end{itemize}
La inducción se la puede ver como una especie de dominó, necesitamos probar que si tiramos el primer dominó, caen todos los demás automáticamente. \\
Lo primero que hacemos es probar un caso base, y luego el paso inductivo. 
\begin{itemize}
    \item Caso Base (CB) (tirar el primer dominó): Acá estamos viendo si con la "fórmula" que queremos probar verdadera para todo natural el primer caso funciona.
    \item Paso Inductivo
    \begin{itemize}
        \item Hipótesis Inductiva (HI): lo que asumo como verdadero, mi "fórmula".
        \item Quiero Probar Que (QPQ): En donde aplico la Hipótesis inductiva para ver si el término siguiente o todos los siguientes siempre son verdaderos.
    \end{itemize}
\end{itemize}
Básicamente, lo que queremos probar es que: $P(1) \implies P(2) \implies P(3) ... \ P(n)$ sean todos verdaderos. Con un caso falso, la proposición es refutada pues quedaría algo como $V \implies F \implies ...$ y como sabemos $V \implies F$ directamente mata todo pues es falso. \\ \\
\textbf{Importante}: En el paso inductivo, la variable que usemos es un \textbf{natural fijo}. Por lo tanto, al terminar la inducción, si es verdadero para k y lo que hayamos querido probar tenemos que predicar sobre n aparte.
\subsection*{Inducción Simple}
Comenzamos probando desde n=1 o n=0. Esto es súper importante porque a partir de n=2 ya es inducción corrida (lo vemos más adelante). \\
\textbf{Caso Base}: P(1) se cumple \\
\textbf{Paso Inductivo}: Sea $ k \in \nat$
\begin{itemize}
    \item \textbf{HI}: P(k) V
    \item \textbf{QPQ}: P(k+1) es V
\end{itemize}
Si llego a lo que quería, es decir, P(k+1) vale entonces P(k) vale y P(n) vale $\forall n \in \nat$ \\ \\
\textbf{Importante}: Notar que la HI y la QPQ hablamos de un k, pero a la hora de llegar a la conclusión hablamos de n. Esto es porque el \textbf{k está fijo} y es nuestro "natural" de laboratorio. \\
Véase \hyperref[subsec:induccion_simple]{\underline{anexo}} para ver ejemplos.
\subsection*{Inducción Doble}
Es exactamente lo mismo que en la Inducción Simple solo que acá tenemos que demostrar que 2 casos base son verdaderos. Luego, tenemos 2 hipótesis inductivas que las tenemos que usar en nuestro qpq.
\subsection*{Inducción Corrida}
Sea $n_{0}$ un $n>1 \in \nat$ \\
Ej.: Si me piden probar desde $n>3$ entonces tengo que considerar los naturales a partir de $n_{0}$. Los $n<n_{0}$ no nos importan. Por lo tanto las implicaciones falsas terminan dando verdadero. \\
\[P(1) \implies P(2) \implies P(3) \implies P(4) \equiv F \implies F \implies F \implies ? \equiv ? \]
Nótese que ? es lo que justamente queremos probar en el caso base, necesitamos tirar a partir del caso P(4) que no sabemos si todavía es verdadero. \\
\textbf{Caso Base}: P($n_{0}$) se cumple con $n_{0}>n$ \\
\textbf{Paso Inductivo}: Sea $ k \in \nat\ge n_{0} $
\begin{itemize}
    \item \textbf{HI}: P(k)
    \item \textbf{QPQ}: P(k+1)
\end{itemize}
\textbf{Importante}: En el paso inductivo no hablo de ninguna restricción de k porque ya está limitado en el caso base. \\
Si llego a lo que quería, es decir, P(k+1) vale entonces P(k) vale $\forall n \in \nat \ge n_{0}$ y P(n) vale $\forall n \in \nat \ge n_{0}$ \\
Véase \hyperref[subsec:induccion_corrida]{\underline{anexo}} para ver ejemplos.
\subsection*{Sucesiones Definidas por Recurrencia}
Hablamos de recurrencia cuando para calcular un nuevo término necesito el/los término/s anterior/es.  \\
Este tipo de inducción son un poco más complicada porque tenemos que buscar una fórmula cerrada, más que nada porque los datos que nos afrontamos son solamente una guía para entender como evoluciona la secuencia a lo largo del tiempo. \\
Las Sucesiones Definidas por Recurrencia tienen esta pinta:
\[\left\{ \begin{array}{lcc} a_{1} = x  \\ a_{n+1} = a_{n} \end{array} \right.\]
La sucesión definida anteriormente tiene muchas variables, por lo tanto no se considera una fórmula cerrada.  \\
Una fórmula cerrada es aquella fórmula que con solo dar una variable, podemos obtener un resultado. \\
\textbf{Caso Base}: P(n) se cumple \\
\textbf{Paso Inductivo}: Sea $ k \in \nat\ge m $
\begin{itemize}
    \item \textbf{HI}: P(k)
    \item \textbf{QPQ}: P(k+1) 
\end{itemize}
Véase \hyperref[subsec:sucesiones_recurrencia]{\underline{anexo}} para ver ejemplos.
\subsection*{Inducción Completa/Global}
Este tipo de inducción consiste en, por cada término que queremos probar, necesitamos ver que efectivamente todos los anteriores valgan. Es decir, si llegamos a estar probando n=10 entonces tenemos que efectivamente verificar de vuelta que n=1, n=2 sean verdaderos. \\
Esto normalmente se da con sucesiones en forma de recurrencia. \\
\textbf{Ej.}: $ a_{1} = 5, a_{n+2} = 5a_{n+1} - 6a_{n}$ debe ser probado con inducción completa pues necesitamos para cada término, sus dos anteriores. \\
\textbf{Importantísimo}: Es posible que haya casos donde podemos aplicar distintos tipos de inducción, es decir, puede que podamos aplicar inducción global en un caso, pero sea más fácil de resolver utilizando inducción simple.
Véase \hyperref[subsec:induccion_completa]{\underline{anexo}} para ver ejemplos.
\subsection*{Inducción Completa/Global Corrida}
Comenzamos probando desde un n>1. Lo demás es igual que la Inducción Completa/Global.
\subsection*{Cantidad de Casos Base}
La cantidad de casos base que necesitamos viene directamente de lo que necesitamos para probar nuestra hipótesis.
\begin{itemize}
    \item En Inducción Doble son dos.
    \item En Inducción Completa/Global depende de la cantidad de términos que se usen para $"tirar"$ el primer dominó.
    \begin{itemize}
        \item $ a_{1} = 5, a_{n+2} = 5a_{n+1} - 6a_{n}$. Necesitamos dos casos base pues el segundo término depende de sus dos anteriores y ambos deben ser verdaderos.
        \item Una sumatoria tipo $ \sum_{i=0}^{n}{a_{i}}$ tiene un solo caso base porque necesitamos solamente tirar el primero para que arranquen los demás, no necesito ningún anterior. Sin embargo, esta inducción es global porque necesito todos los anteriores en cada paso.
    \end{itemize}
\end{itemize}
\subsection*{Desigualdades}
Acá dejo algunas cosas comúnes que se van a utilizar en inducción por desigualdades.
\begin{itemize}
    \item Se puede pasar lo que yo quiera de izquierda a derecha.
    \item Solo cambia el sentido de la desigualdad si multiplico por -1.
    \item Siempre debo dejar expresiones simples a comparar
    \begin{itemize}
        \item $k^{2} + k + k \ge k+1$, es muy dificil de leer.
        \item $k^{2} \ge 0$ es más fácil de leer y dice lo mismo que la de arriba.
    \end{itemize}
\end{itemize}
\section*{Combinatoria}
El arte de contar. Un tema bueno, pero que no se le dedica el suficiente tiempo como para asimilar las cosas.
\begin{itemize}
    \item Unión de Conjuntos Disjuntos: $\#A\cup B = \#A + \#B $. Como no tienen elementos en común (intersección vacía), no estamos sumando ningún repetido. 
    \item Principio Inclusión Exclusión: $\#A\cup B = \#A + \#B - \#A\cap B  $. Como A y B tienen elementos en común, eliminamos los repetidos.
    \item Diferencia de Conjunto Incluido en otro: $\#A-B = \#A-\#B$. Acá B está dentro de A, por lo tanto, eliminamos los elementos de B de A.
    \item Diferencia: $\#A-B = \#A - \#A\cap B$. Nos quedamos solo con los elementos de A
    \item Diferencia Simétrica: $\#(A\triangle B) = \#A + \#B - 2\#(A \cap B)$
    \item Total de casos independientes: $\#(AxB) = \#A \ast \#B $ 
\end{itemize}
\textbf{Importantísimo}: Cuando hacemos cálculos con cardinales, lo más cómodo es quedarnos solamente con las intersecciones pues es más fácil de manipular. Esto es porque la unión es verdadera en varios casos, está en A o está en B o en ambos. En la intersección solo es verdadera si está en ambas.
\subsection*{Casos sin condiciones}
\textbf{Ej. 1}: ¿Cuántos pares puedo armar con $ A = \{2, 4\}, B = \{1, 3\}$? \\
Básicamente lo que nos pide el ejemplo es ver cuantas combinaciones entre los elementos de A y B hay. \\
Lo que podemos saber es que tengo que armar pares que puedan ser de la forma (A, A), (A, B), (B, A), (B, B) \\
¿Cuantas posibilidades, para cada elemento tenemos en cada lugar? 4 en cada uno. \\
Entonces, el cálculo sale de hacer (4, 4), que esto lo traducimos a $ 4 \ast 4 = 16$. Entonces hay 16 posibles pares con los elementos de A y B. \\

\textbf{Ej. 2}: Mismo caso que el anterior, pero que ahora solo los pares sean de la forma (A, B). \\
Si los pares ahora son de la forma (A, B) son parecidos al producto cartesiano, es decir, voy a tener (2, 2) posibilidades. \\
Entonces, el cálculo sale de hacer $2 \ast 2 = 4$. Entonces hay 4 posibles pares, y estos son: $\{(2, 1), (2, 3), (4, 1), (4, 3)\}$ \\

En este tipo de ejemplos hablamos de cálculos \textbf{sin condiciones} porque utilizamos todo el conjunto de valores.
\subsection*{Casos con condiciones}
\textbf{Ej. 1}: Dado $A = \{1, ..., 10\}$ calcule la cantidad de elementos de B sabiendo que $ B = \{(a, b) \in AxA, a \neq b\}$ \\
En este caso, podemos ver que si bien vamos a utilizar todos los valores de A, seguramente haya combinaciones que no nos sirvan. B no va a tener pares donde el primer elemento sea igual al segundo. \\
Por lo tanto, habría que excluir casos como: $(1, 1), (2, 2), (3, 3) \ ...\ (10, 10)$ \\
En este caso es fácil, porque como los números son acotados (10) sabemos que pares repetidos va a haber 10, es decir, 1 por cada número. \\
Entonces la idea sería, que hay $(10, 10) - 10$ pero ¿de donde salió esto?, sale de considerar que va a haber pares de números que puedan tener los 10 posibles valores de A, y luego, se restan la cantidad de posibilidades de que salgan repetidos. \\
Entonces, el cálculo sale de hacer $10 \ast 10 = 100 - 10 = 90$. \\
Este tipo de casos, lo podemos generalizar diciendo que $"gastamos"$ uno de los valores, luego, quedaría $(10 ops, 9 ops) = 9 \ast 10$. Como estamos reduciendo las opciones para el segundo lugar, estamos garantizando que no esté el mismo valor que estaba en el primer lugar.
\subsection*{Casos donde calculamos un total, en base a varios grupos}
\textbf{Ej. 1}: Dado $A = \{1, ..., 10\}$ calcule la cantidad de subconjuntos en C sabiendo que $ C = \{(a, b) \in AxA; \ a > b\}$ \\
En este caso, podemos ver que los pares que hay que armar nos dicen que tenemos que la primera coordenada debe ser mayor a la segunda. Con un análisis rápido podemos ver que el grupo que tendrá mas elementos será el del 10, porque el 10 es mayor que el 1, 2, 3, 4, 5, 6, 7, 8 y el 9. \\
Los casos serían algo así (1, 0 op. disp), (2, 1 op. disp), (3, 2 op. disp), ... (10, 9 op. disp) \\
Entonces, el cálculo sale de hacer $\frac{9 \ast 10}{2}$ donde 9 es la cantidad de grupos de elementos que tendremos (nótese que el 1 no lo contamos xq no es mayor que ninguno) y el 10 es la cantidad de elementos que tenemos para elegir. Es importante tambien notar, que estamos dividiendo por 2 porque los casos como $(2, 1) = (1, 2)$ en conjuntos son idénticos. 
\subsection*{Permutaciones}
Sea A que tiene n elementos, una permutación es un conjunto de longitud n ordenado distinto que A. Básicamente, cambiar el orden de los elementos. \\
$Ej.$: $ A = \{1, 2, 3\}$ las permutaciones de A son $\{1, 3, 2\}, \{2, 1, 3\}, \{2, 3, 1\}, \{3, 1, 2\}, \{3, 2, 1\}$ 
\section*{Anexo}
\subsection*{Pertenencia en Conjuntos}
\label{subsec:pertenecencia_conjuntos}
Sea A el conjunto: $\{1, 2, \{C, B\}, F, \{10, 15\}\}$
\begin{itemize}
    \item $ 1 \in A, 2 \in A, F \in A $
    \item $ C \notin A, B \notin A $
    \item $  \{C, B\}, \{10, 15\} \in A $
\end{itemize}
¿Por qué $C \notin A$? Pues C no es un elemento de A.\\ Notar que C es parte del elemento $\{C, B\}$ en A, pero C no es un elemento independiente.
\subsection*{Inclusión en Conjuntos}
\label{subsec:inclusion_conjuntos}
\textbf{Ex. 1}: Sea $A = \{1, 2, 3\} \ y \ D = \{1, 3\}$. ¿Es D un subconjunto de A? \\
Sí, lo es pues $1 \in A$ y $ 3 \in A$ \\
\textbf{Ex. 2}: Sea $A = \{1, \{1, 4\}, 3, 10\}$
\begin{itemize}
    \item $ \{1, 4\} \nsubseteq A $ pues no existen 1 y 4 como elementos en A
    \item $ \{1, 4\} \in A $ pues $\{1, 4\}$ es un elemento de A
    \item $\{1, 3\} \subseteq A$ pues $ 1 \in A, 3 \in A$, lo mismo sucede con $ \{1, 10\} \ o \ \{3, 10\}$
\end{itemize}
\subsection*{Cuantificadores}
\label{subsec:cuantificadores}
\textbf{Ex. 1}: $A = \{2, 4, 6, 8\}$ \\
Algunos ejemplos utilizando cuantificadores 
\begin{itemize}
    \item $\forall x \in A \ \symbol{92} \ x \% 2 = 0$ (Todos pares en A)
    \item $\neg \ \exists x \in A \ \symbol{92} \ x \% 2 \neq 0$ (No existe ningún impar en A) 
    \item $ \exists x \in A \ \symbol{92} x = 4$ (Existe un elemento en A que es exactamente 4)
\end{itemize}
\subsection*{Inducción Simple}
\label{subsec:induccion_simple}
\subsection*{Inducción Corrida}
\label{subsec:induccion_corrida}
\subsection*{Sucesiones Definidas por Recurrencia}
\label{subsec:sucesiones_recurrencia}
\subsection*{Inducción Completa/Global}
\label{subsec:induccion_completa}
\end{document} 

