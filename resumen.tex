\documentclass[10pt,a4paper]{article}
\usepackage{blindtext}
\usepackage{subcaption}
\usepackage{graphicx}
\usepackage{tikz}
\usepackage{amssymb}
\usepackage{caption}
\usepackage{amsmath}
\usepackage{circuitikz}
\usepackage{hyperref}
\usepackage{amssymb}
\usepackage{amsmath}
\usepackage{listings}

\lstset{
    inputencoding=utf8,
    extendedchars=true,
    literate={á}{{\'a}}1 {é}{{\'e}}1 {í}{{\'i}}1 {ó}{{\'o}}1 {ú}{{\'u}}1 {ñ}{{\~n}}1 {Á}{{\'A}}1 {É}{{\'E}}1 {Í}{{\'I}}1 {Ó}{{\'O}}1 {Ú}{{\'U}}1 {Ñ}{{\~N}}1
}
\input{AEDmacros}
\newcommand{\notimplies}{\;\not\!\!\!\implies}
\title{Álgebra I}
\author{Tomás Agustín Hernández}
\date{}

\begin{document}
\maketitle

\begin{figure}[b]
    \centering
    \begin{tikzpicture}[remember picture,overlay]
        \node[anchor=south east, inner sep=0pt, xshift=-1cm, yshift=2cm] at (current page.south east) {
            \begin{minipage}[b]{0.5\textwidth}
                \includegraphics[width=\linewidth]{logo_uba.jpg}
                \label{fig:bottom}
            \end{minipage}
        };
    \end{tikzpicture}
\end{figure}

\newpage
\section*{Conjuntos}
Los conjuntos almacenan elementos, no se consideran repetidos y responde a la pregunta de $"$¿está el elemento?$"$, esto último quiere decir que no tenemos forma de tomar un elemento sino predicar acerca de si está o no.
\subsection*{Pertenecencia a un Conjunto}
Si consideramos cualquier elemento $x$, decimos que está en un conjunto A si \textbf{x pertenece a A}. \\
Se denota como: $x \in A$ \\
\textbf{Importante}: La relación está dada por $Elemento \ en \ Conjunto$ \\
Véase \hyperref[subsec:pertenecencia_conjuntos]{\underline{ánexo}} para ejemplos más didácticos.

\section*{Anexo}
\subsection*{Pertenencia en Conjuntos}
\label{subsec:pertenecencia_conjuntos}
Sea A el conjunto: $\{1, 2, \{C, B\}, F, \{10, 15\}\}$
\begin{itemize}
    \item $ 1 \in A, 2 \in A, F \in A $
    \item $ C \notin A, B \notin A $
    \item $  \{C, B\}, \{10, 15\} \in A $
\end{itemize}
¿Por qué $C \notin A$? Pues C no es un elemento de A.\\ Notar que C es parte del elemento $\{C, B\}$ en A, pero C no es un elemento independiente.

\end{document} 
