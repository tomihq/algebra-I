\documentclass[10pt,a4paper]{article}
\usepackage{blindtext}
\usepackage{subcaption}
\usepackage{graphicx}
\usepackage{tikz}
\usepackage{amssymb}
\usepackage{caption}
\usepackage{amsmath}
\usepackage{circuitikz}
\usepackage{hyperref}
\usepackage{amssymb}
\usepackage{amsmath}
\usepackage{listings}

\lstset{
    inputencoding=utf8,
    extendedchars=true,
    literate={á}{{\'a}}1 {é}{{\'e}}1 {í}{{\'i}}1 {ó}{{\'o}}1 {ú}{{\'u}}1 {ñ}{{\~n}}1 {Á}{{\'A}}1 {É}{{\'E}}1 {Í}{{\'I}}1 {Ó}{{\'O}}1 {Ú}{{\'U}}1 {Ñ}{{\~N}}1
}
\input{AEDmacros}
\newcommand{\notimplies}{\;\not\!\!\!\implies}
\title{Álgebra I}
\author{Tomás Agustín Hernández}
\date{}

\begin{document}
\maketitle

\begin{figure}[b]
    \centering
    \begin{tikzpicture}[remember picture,overlay]
        \node[anchor=south east, inner sep=0pt, xshift=-1cm, yshift=2cm] at (current page.south east) {
            \begin{minipage}[b]{0.5\textwidth}
                \includegraphics[width=\linewidth]{logo_uba.jpg}
                \label{fig:bottom}
            \end{minipage}
        };
    \end{tikzpicture}
\end{figure}

\newpage
\section*{Conjuntos}
Los conjuntos almacenan elementos, \textbf{no se consideran repetidos ni tampoco importa el orden}. Responde a la pregunta de $"$¿está el elemento?$"$, esto último quiere decir que no tenemos forma de tomar un elemento sino predicar acerca de si está o no. \\
\textbf{Ej.}: $ A = \{1, 2\}, B = \{2, 1\}$. A y B son considerados iguales, pues no importa el orden sino los elementos que tienen dentro.
\subsection*{Pertenecencia a un Conjunto}
Si consideramos cualquier elemento $x$, decimos que está en un conjunto A si \textbf{x pertenece a A}. \\
La pertenencia de un elemento a un conjunto la denotamos como: $x \in A$ \\
\textbf{Importante}: La relación está dada por $Elemento \ en \ Conjunto$ \\
Véase \hyperref[subsec:pertenecencia_conjuntos]{\underline{ánexo}} para ejemplos más didácticos.
\subsection*{Inclusión a un Conjunto}
Sean A y D conjuntos cualesquiera.
Decimos que D es un subconjunto de A sí y solo sí todos los elementos de D están en A. \\
La inclusión en un conjunto la denotamos como $D \subseteq A$ \\
Es posible leer el símbolo $ \subseteq $ de tres maneras: 
\begin{itemize}
    \item "D es un subconjunto de A"
    \item "D está incluido en A"
    \item "D está contenido en A"
\end{itemize}
Los subconjuntos posibles no salen más que haciendo combinaciones con sus elementos, es decir, agruparlos de diferentes formas. \\
Véase \hyperref[subsec:inclusion_conjuntos]{\underline{ánexo}} para ejemplos más didácticos.
\subsection*{Cardinal de un Conjunto}
Sea A un conjunto, el cardinal de un conjunto indica la cantidad de elementos en el conjunto. \\
Se denota como: $\#A$
\subsection*{Cantidad de Subconjuntos posibles dado un Conjunto}
Sea un conjunto A, la cantidad de subconjuntos D para el conjunto A es: $2^{\#A}$
\subsection*{Elemento Vacío}
Se representa con el símbolo de $\emptyset$. El elemento vacío está \textbf{incluido} en todos los conjuntos. \\
\textbf{Importante}: El elemento vacío NO pertenece a todos los conjuntos sino que está incluido en todos.
\subsection*{Cuantificadores}
Nos permiten predicar acerca de los elementos de un conjunto dado. 
\begin{itemize}
    \item $\forall$ x: Para todo x.
    \begin{itemize}
        \item Para que sea verdadero todos deben cumplir la condición dada.
        \item Es falso si existe un caso en que no se cumple.
    \end{itemize}
    \item $\exists$ x: Existe un x
    \begin{itemize}
        \item Para que sea verdadero alcanza con encontrar un caso verdadero.
        \item Es falso si no hay ningun caso que cumpla la condición
    \end{itemize}
\end{itemize}
\textbf{Importante}: El símbolo de $:$ o $\symbol{92}$ significa "tal que" \\
Véase \hyperref[subsec:cuantificadores]{\underline{ánexo}} para ejemplos más didácticos.
\section*{Operaciones entre Conjuntos}
Sean A y B conjuntos cualesquiera. La cantidad de filas que tendrá una tabla de verdad es: \textbf{$2^{cantVariables}$} \\
\textbf{Importante}: Las operaciones entre conjuntos que vamos a ver están relacionadas con la lógica proposicional.
\newpage
\subsection*{Unión $(A \cup B)$}
Es exactamente igual como en la lógica proposicional. La unión es un $o$ lógico. En el conjunto resultante quedan los elementos de A y B. \\

\begin{table}[h!]
    \centering
    \begin{tabular}{|c | c | c|}
    \hline
    \textbf{A} & \textbf{B} & \textbf{$A \cup B$} \\[0.1cm]
    \hline
    V & V & V \\
    V & F & V \\
    F & V & V \\
    F & F & F \\
    \hline
    \end{tabular}
    \caption{Unión de conjuntos}
\end{table} 
Cada fila se puede generalizar para un x cualquiera en las operaciones lógicas. \\
\textbf{Ej.}: Si $x \in A \land x \in B$ entonces $ x \in A \cup B$ esto claramente nos dice que estamos en el caso de la fila 1. \\
\textbf{Ej.}: Si $x \notin A \land x \in B$ entonces $ x \in A \cup B$ esto claramente nos dice que estamos en el caso de la fila 3.
\subsection*{Intersección $(A \cap B)$}
Es exactamente igual como en la lógica proposicional. La intersección es un $"y"$ lógico. En el conjunto resultante quedan los elementos que están tanto en A y en B.
\begin{table}[h!]
    \centering
    \begin{tabular}{|c | c | c|}
    \hline
    \textbf{A} & \textbf{B} & \textbf{$A \cap B$} \\[0.1cm]
    \hline
    V & V & V \\
    V & F & F \\
    F & V & F \\
    F & F & F \\
    \hline
    \end{tabular}
    \caption{Intersección de conjuntos}
\end{table} \\
Cada fila se puede generalizar para un x cualquiera en las operaciones lógicas. \\
\textbf{Ej.}: Si $x \in A \land x \in B$ entonces $ x \in A \cap B$ esto claramente nos dice que estamos en el caso de la fila 1. \\
\textbf{Ej.}: Si $x \notin A \land x \in B$ entonces $ x \notin A \cap B$ esto claramente nos dice que estamos en el caso de la fila 3.
\subsection*{Complemento $(A \cap B)$}
En la lógica proposicional, el complemento es la negación. Lo que está en un conjunto universal V pero no en el conjunto.
\begin{table}[h!]
    \centering
    \begin{tabular}{|c | c|}
    \hline
    \textbf{A} & \textbf{$\neg A$} \\[0.1cm]
    \hline
    V & F \\
    V & F  \\
    F & V  \\
    F & V \\
    \hline
    \end{tabular}
    \caption{Complemento en Conjuntos}
\end{table} \\
Cada fila se puede generalizar para un x cualquiera en las operaciones lógicas. \\
\textbf{Ej.}: Si $x \in A$ entonces termina siendo $ x \notin A$ esto claramente nos dice que estamos en el caso de la fila 1. \\

Sea $A=\{1, 2\}, B=\{3, 4, 5\}, C=\{8, 9\}, V=\{A, B, C\} \implies A^{c} = \{3, 4, 5, 8, 9\}$ \\
\textbf{Importante}: Nótese que siempre se hace el complemento en base a los elementos que hay en el universo y se excluyen algunos. En este caso, del universo V nos quedamos con los que NO están en A.
\newpage
\subsection*{Diferencia $(A-B )$}
Esta operación es conocida también de la siguiente manera $A\symbol{92}B$.
Es una equivalencia de $A \cap B^{c}$. Representa lo que está en A pero no en B. Si se lo quisiera representar en la tabla de verdad, debe representar la equivalencia.
\begin{table}[h!]
    \centering
    \begin{tabular}{|c | c | c | c|}
    \hline
    \textbf{A} & \textbf{B} & \textbf{$B^{c}$} & \textbf{$A \cap B^{c}$} \\[0.1cm]
    \hline
    V & V & F & F\\
    V & F & V & V\\
    F & V & F & F\\
    F & F & V & F\\
    \hline
    \end{tabular}
    \caption{Diferencia de conjuntos}
\end{table} \\
\subsection*{Diferencia Simétrica $(A \triangle B)$}
Equivalente al XOR($\veebar$) u $o$ excluyente en la lógica proposicional. \\
Es una equivalencia de $(A - B) \cup (B - A)$ y $(A\cup B) - (A \cap B)$. Representa lo que está en A o en B pero no en ambos. 
\begin{table}[h!]
    \centering
    \begin{tabular}{|c | c | c | c | c|}
    \hline
    \textbf{A} & \textbf{B} & \textbf{$A \veebar B$} & \textcolor{blue}{$(A - B) \cup (B - A)$} & \textcolor{blue}{$(A\cup B) - (A \cap B)$} \\[0.1cm]
    \hline
    V & V & F & F & F \\
    V & F & V & V & V \\
    F & V & V & V & V \\
    F & F & F & V & V \\
    \hline
    \end{tabular}
    \caption{Diferencia Simétrica en conjuntos}
\end{table} \\
\textbf{Nota}: Las columnas en azul son equivalencias a la operación $\veebar$ y son útiles a la hora de demostrar. \\
\textbf{Ej.}: Si $x \in A \land x \in B$ entonces $ A \veebar B = F$ esto claramente nos dice que estamos en el caso de la fila 1.
\textbf{Ej.}: Si $x \in A \land x \notin B$ entonces $ A \veebar B = V$ esto claramente nos dice que estamos en el caso de la fila 2. \\
\subsection*{Inclusión $(A \subseteq B)$}
Representa el $\implies$ de la lógica proposicional. Recordemos que la inclusión es verdadera si todos los elementos de A están en B siendo A y B conjuntos cualesquiera. \\
Es lo que vamos a utilizar para demostrar, y es importante que se lo entienda bien. 
\begin{table}[h!]
    \centering
    \begin{tabular}{|c | c | c|}
    \hline
    \textbf{A} & \textbf{B} & \textbf{$A \implies B$} \\[0.1cm]
    \hline
    V & V & V \\
    V & F & F \\
    F & V & V \\
    F & F & V \\
    \hline
    \end{tabular}
    \caption{Inclusión de conjuntos}
\end{table} 
\begin{itemize}
    \item El único caso que nos importa es que si el antecedente es verdadero, hay que ver que el consecuente NO sea falso. En las demostraciones asumimos que vale el antecedente y tenemos que ver si hace verdadero al consecuente.
    \item Si no se cumple el antecedente, el consecuente es siempre verdadero.
\end{itemize}
Cada fila se puede generalizar para un x cualquiera en las operaciones lógicas. \\
\textbf{Ej.}: Sea $A = \{1, 2, 3\} \ B = \{10, 40\} \ x = 100$ ¿Se cumple que $ x \in A \implies x \in B$? ¿100 está en A? No, y al ser una implicación si el antecedente no se cumple, queda toda la proposición verdadera. Luego, sí, se cumple que $ x \in A \implies x \in B$. Esto claramente nos dice que estamos en el caso de la fila 3. \\ 
\textbf{Ej.}: Sea $A = \{1, 2, 3\} \ B = \{10, 40\} \ x = 3$ ¿Se cumple que $ x \in A \implies x \in B$? ¿3 está en A? Sí. Entonces esto hace al antecedente verdadero ¿me basta para decir que la proposición es verdadera? No. Primero debo ver qué pasa con el consecuente. ¿Es cierto que 3 está en B? No. Entonces como el antecedente es verdadero y el consecuente es falso, la proposición es falsa. Luego, no, no se cumple que $ x \in A \implies x \in B$. Esto claramente nos dice que estamos en el caso de la fila 2. \\
\textbf{Nota}: Que se entiendan los ejemplos anteriormente mencionados es realmente importante. Se usa en prácticamente todas las demostraciones.\subsection*{Igualdad $(A \iff B)$}
Representa el $\iff$ (sí y solo sí) de la lógica proposicional. Recordemos que la igualdad es verdadera si todos los elementos de A están en B siendo A y B conjuntos cualesquiera. \\
\begin{table}[h!]
    \centering
    \begin{tabular}{|c | c | c|}
    \hline
    \textbf{A} & \textbf{B} & \textbf{$A \iff B$} \\[0.1cm]
    \hline
    V & V & V \\
    V & F & F \\
    F & V & F \\
    F & F & F \\
    \hline
    \end{tabular}
    \caption{Igualdad de conjuntos}
\end{table} 
\begin{itemize}
    \item La manera de demostrar esto es viendo si se cumple que $A \subseteq B \ y \ B \subseteq A$
\end{itemize}
Cada fila se puede generalizar para un x cualquiera en las operaciones lógicas. 
\subsection*{Leyes de De Morgan}
La forma más fácil de verlo es que se distribuye el complemento y se invierte la operación.
\begin{itemize}
    \item $(A \cup B)^{c} = A^{c} \cap B^{c}$
    \item $(A \cap B)^{c} = A^{c} \cup B^{c}$
\end{itemize}
\subsection*{Propiedades de Conjuntos}
\begin{itemize}
    \item Distributiva: $A \cap (B \cup C) = (A \cap B) \cup (A \cap C)$
    \item Conmutatividad: $A \cap B = B \cap A $ Igual para unión
    \item Conjuntos Disjuntos $ A \cap B = \emptyset$
\end{itemize}
TODO: Agregar en Anexo demostración de distributividad.
\subsection*{Producto Cartesiano ($A X B$)}
Sean dos conjuntos A y B cualquiera. El producto cartesiano es el par ordenado (c, d) con $c \in A$ y $d \in B$. \\ 
La cantidad de elementos máxima en un producto cartesiano es = $\#A \ast \#B$. \\
Sí o sí es necesario que el par NO sea nulo, es decir, deben ser elementos válidos. \\
\textbf{Importante}: AXB $\neq$ BXA \\
\textbf{Ej.}: $A = \{1, 2, 3\}, B = \emptyset, AXB = \emptyset$, pues B está vacío. \\
\textbf{Ej.}: $A = \{1, 2, 3\}, B = \{4, 5\}, \#AXB = 6, AXB = \{\{1, 4\}, \{1, 5\}, \{2, 4\}, \{2, 5\}, \{3, 4\}, \{3, 5\} \}$
\section*{Relaciones}
Sean A y B conjuntos. Una relación A en B es un subconjunto cualquiera R de AXB. \\
\textbf{Ej.}: $A = \{1\}, B = \{4, 5\}, AXB=\{\{1, 4\}, \{1, 5\}\}, R = \{\{1, 1\}, \{1, 2\}, \{1, 4\}\}$ ¿Es R una relación válida de AXB? No, no lo es pues $\{1, 1\} \in R$ pero $ \{1, 1\} \notin AXB$
\section*{Relaciones de un conjunto en sí mismo}
Sea A un conjunto cualquiera. Se dice que A está relacionado con A sí y solo sí AXA. \\
Se dice que R es una relación en A cuando $ R \subseteq AXA$ \\
\textbf{Ej.}: $A = \{1, 2, 3\}, AXA=\{\{1, 1\}, \{1, 2\}, \{1, 3\}, \{2, 2\}, \{2, 3\}, \{3, 3\}\}, R = \{\{1, 2\}, \{1, 4\}\}$ ¿Es R una relación válida de AXB? No, no lo es pues $\{1, 4\} \in R$ pero $ \{1, 4\} \notin AXB$ \\
\textbf{Ej.}: $A = \{1, 2, 3\}, AXA=\{\{1, 1\}, \{1, 2\}, \{1, 3\}, \{2, 2\}, \{2, 3\}, \{3, 3\}\}, R = \{\{1, 1\}, \{1, 2\}, \{1, 3\}\}$ ¿Es R una relación válida de AXB? Sí lo es, pues todos los subconjuntos pertenecientes a R pertenecen a AXA. \\
Veamos ahora \textbf{las propiedades de las relaciones de un conjunto en sí mismo}.
\subsection*{Reflexividad}
Una relación es reflexiva sí y solo sí para todo elemento de A, a está relacionado con A. \\
\textbf{Formalmente}: $ \forall a \in A \implies aRa$ \\
\textbf{Ej.}: $A = \{1, 2, 3\}, AXA=\{\{1, 1\}, \{1, 2\}, \{1, 3\}, \{2, 2\}, \{2, 3\}, \{3, 3\}\}, R = \{\{1, 1\}, \{1, 2\}, \{1, 3\}\}$ ¿Es R una relación válida de AXB? Sí lo es. ¿Es reflexiva? No, no lo es, pues 2 no está relacionado con 2, ni tampoco 3 con el 3.\\
\textbf{Ej.}: $A = \{1, 2, 3\}, AXA=\{\{1, 1\}, \{1, 2\}, \{1, 3\}, \{2, 2\}, \{2, 3\}, \{3, 3\}\}, R = \{\{1, 1\}, \{2, 2\}, \{3, 3\}, \{1, 2\}\}$ ¿Es R una relación válida de AXB? Sí lo es. ¿Es reflexiva? Sí, pues para todo elemento a en R, aRa.\\

\textbf{Nota}: Una relación que solamente tiene dentro los elementos aRa es llamada identidad. Considerando el AXA anterior, la relación identidad sería R = $\{\{1, 1\}, \{2, 2\}, \{3, 3\}\}$
\subsection*{Simetría}
Sean a, b $\in$ A. Una relación es simétrica sí y solo sí $aRb \implies bRa$. Vulgarmente decimos que si uno está relacionado con el otro, el otro está obligado a estarlo también. \\
\textbf{Formalmente}: $ \forall a \in A \implies aRa$ \\
\textbf{Ej.}: $R = \{\{1, 2\}, \{3, 1\}\}$, no es simétrica pues sucede que $1R2$ pero 2 no está relacionado con 1. \\
\textbf{Ej.}: $R = \{\{1, 2\}, \{2, 1\}\}$, es simétrica pues para todo elemento relacionado, se relacionan conjuntamente. \\
\textbf{Ej.}: $R = \{\{1, 1\}\}$, es simétrica, pues no existe ninguna relación entre elementos diferentes. Por lo tanto, el antecedente es falso, luego la proposición ($aRb \implies bRa$) es verdadera \\

\textbf{Nota}: Como es una implicación, si el antecedente es falso (no hay ningún elemento, o no existe relación entre ellos) entonces es simétrica.
\section*{Anexo}
\subsection*{Pertenencia en Conjuntos}
\label{subsec:pertenecencia_conjuntos}
Sea A el conjunto: $\{1, 2, \{C, B\}, F, \{10, 15\}\}$
\begin{itemize}
    \item $ 1 \in A, 2 \in A, F \in A $
    \item $ C \notin A, B \notin A $
    \item $  \{C, B\}, \{10, 15\} \in A $
\end{itemize}
¿Por qué $C \notin A$? Pues C no es un elemento de A.\\ Notar que C es parte del elemento $\{C, B\}$ en A, pero C no es un elemento independiente.
\subsection*{Inclusión en Conjuntos}
\label{subsec:inclusion_conjuntos}
\textbf{Ex. 1}: Sea $A = \{1, 2, 3\} \ y \ D = \{1, 3\}$. ¿Es D un subconjunto de A? \\
Sí, lo es pues $1 \in A$ y $ 3 \in A$ \\
\textbf{Ex. 2}: Sea $A = \{1, \{1, 4\}, 3, 10\}$
\begin{itemize}
    \item $ \{1, 4\} \nsubseteq A $ pues no existen 1 y 4 como elementos en A
    \item $ \{1, 4\} \in A $ pues $\{1, 4\} es un elemento de A$
    \item $\{1, 3\} \subseteq A$ pues $ 1 \in A, 3 \in A$, lo mismo sucede con $ \{1, 10\} \ o \ \{3, 10\}$
\end{itemize}
\subsection*{Cuantificadores}
\label{subsec:cuantificadores}
\textbf{Ex. 1}: $A = \{2, 4, 6, 8\}$ \\
Algunos ejemplos utilizando cuantificadores 
\begin{itemize}
    \item $\forall x \in A \ \symbol{92} \ x \% 2 = 0$ (Todos pares en A)
    \item $\neg \ \exists x \in A \ \symbol{92} \ x \% 2 \neq 0$ (No existe ningún impar en A) 
    \item $ \exists x \in A \ \symbol{92} x = 4$ (Existe un elemento en A que es exactamente 4)
\end{itemize}

\end{document} 

