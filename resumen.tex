\documentclass[10pt,a4paper]{article}
\usepackage{blindtext}
\usepackage{subcaption}
\usepackage{graphicx}
\usepackage{tikz}
\usepackage{amssymb}
\usepackage{caption}
\usepackage{amsmath}
\usepackage{circuitikz}
\usepackage{hyperref}
\usepackage{amssymb}
\usepackage{amsmath}
\usepackage{listings}

\lstset{
    inputencoding=utf8,
    extendedchars=true,
    literate={á}{{\'a}}1 {é}{{\'e}}1 {í}{{\'i}}1 {ó}{{\'o}}1 {ú}{{\'u}}1 {ñ}{{\~n}}1 {Á}{{\'A}}1 {É}{{\'E}}1 {Í}{{\'I}}1 {Ó}{{\'O}}1 {Ú}{{\'U}}1 {Ñ}{{\~N}}1
}
\input{AEDmacros}
\newcommand{\notimplies}{\;\not\!\!\!\implies}
\title{Álgebra I}
\author{Tomás Agustín Hernández}
\date{}

\begin{document}
\maketitle

\begin{figure}[b]
    \centering
    \begin{tikzpicture}[remember picture,overlay]
        \node[anchor=south east, inner sep=0pt, xshift=-1cm, yshift=2cm] at (current page.south east) {
            \begin{minipage}[b]{0.5\textwidth}
                \includegraphics[width=\linewidth]{logo_uba.jpg}
                \label{fig:bottom}
            \end{minipage}
        };
    \end{tikzpicture}
\end{figure}

\newpage
\section*{Conjuntos}
Los conjuntos almacenan elementos, no se consideran repetidos y responde a la pregunta de $"$¿está el elemento?$"$, esto último quiere decir que no tenemos forma de tomar un elemento sino predicar acerca de si está o no.
\subsection*{Pertenecencia a un Conjunto}
Si consideramos cualquier elemento $x$, decimos que está en un conjunto A si \textbf{x pertenece a A}. \\
La pertenencia de un elemento a un conjunto la denotamos como: $x \in A$ \\
\textbf{Importante}: La relación está dada por $Elemento \ en \ Conjunto$ \\
Véase \hyperref[subsec:pertenecencia_conjuntos]{\underline{ánexo}} para ejemplos más didácticos.
\subsection*{Inclusión a un Conjunto}
Sean A y D conjuntos cualesquiera.
Decimos que D es un subconjunto de A sí y solo sí todos los elementos de D están en A. \\
La inclusión en un conjunto la denotamos como $D \subseteq A$ \\
Es posible leer el símbolo $ \subseteq $ de tres maneras: 
\begin{itemize}
    \item "D es un subconjunto de A"
    \item "D está incluido en A"
    \item "D está contenido en A"
\end{itemize}
Los subconjuntos posibles no salen más que haciendo combinaciones con sus elementos, es decir, agruparlos de diferentes formas. \\
Véase \hyperref[subsec:inclusion_conjuntos]{\underline{ánexo}} para ejemplos más didácticos.
\subsection*{Cardinal de un Conjunto}
Sea A un conjunto, el cardinal de un conjunto indica la cantidad de elementos en el conjunto. \\
Se denota como: $\#A$
\subsection*{Cantidad de Subconjuntos posibles dado un Conjunto}
Sea un conjunto A, la cantidad de subconjuntos D para el conjunto A es: $2^{\#A}$
\subsection*{Elemento Vacío}
Se representa con el símbolo de $\emptyset$. El elemento vacío está \textbf{incluido} en todos los conjuntos. \\
\textbf{Importante}: El elemento vacío NO pertenece a todos los conjuntos sino que está incluido en todos.
\subsection*{Cuantificadores}
Nos permiten predicar acerca de los elementos de un conjunto dado. 
\begin{itemize}
    \item $\forall$ x: Para todo x.
    \begin{itemize}
        \item Para que sea verdadero todos deben cumplir la condición dada.
        \item Es falso si existe un caso en que no se cumple.
    \end{itemize}
    \item $\exists$ x: Existe un x
    \begin{itemize}
        \item Para que sea verdadero alcanza con encontrar un caso verdadero.
        \item Es falso si no hay ningun caso que cumpla la condición
    \end{itemize}
\end{itemize}
\textbf{Importante}: El símbolo de $:$ o $\symbol{92}$ significa "tal que"
Véase \hyperref[subsec:cuantificadores]{\underline{ánexo}} para ejemplos más didácticos.
\section*{Operaciones entre Conjuntos}
Sean A y B conjuntos cualesquiera.
\subsection*{Unión $(A \cup B)$}
Es exactamente igual como en la lógica proposicional. La unión es un "ó" lógico.
\section*{Anexo}
\subsection*{Pertenencia en Conjuntos}
\label{subsec:pertenecencia_conjuntos}
Sea A el conjunto: $\{1, 2, \{C, B\}, F, \{10, 15\}\}$
\begin{itemize}
    \item $ 1 \in A, 2 \in A, F \in A $
    \item $ C \notin A, B \notin A $
    \item $  \{C, B\}, \{10, 15\} \in A $
\end{itemize}
¿Por qué $C \notin A$? Pues C no es un elemento de A.\\ Notar que C es parte del elemento $\{C, B\}$ en A, pero C no es un elemento independiente.
\subsection*{Inclusión en Conjuntos}
\label{subsec:inclusion_conjuntos}
\textbf{Ex. 1}: Sea $A = \{1, 2, 3\} \ y \ D = \{1, 3\}$. ¿Es D un subconjunto de A? \\
Sí, lo es pues $1 \in A$ y $ 3 \in A$ \\
\textbf{Ex. 2}: Sea $A = \{1, \{1, 4\}, 3, 10\}$
\begin{itemize}
    \item $ \{1, 4\} \nsubseteq A $ pues no existen 1 y 4 como elementos en A
    \item $ \{1, 4\} \in A $ pues $\{1, 4\} es un elemento de A$
    \item $\{1, 3\} \subseteq A$ pues $ 1 \in A, 3 \in A$, lo mismo sucede con $ \{1, 10\} \ o \ \{3, 10\}$
\end{itemize}
\subsection*{Cuantificadores}
\label{subsec:cuantificadores}
\textbf{Ex. 1}: $A = \{2, 4, 6, 8\}$ \\
Algunos ejemplos utilizando cuantificadores 
\begin{itemize}
    \item $\forall x \in A \ \symbol{92} \ x \% 2 = 0$ (Todos pares en A)
    \item $\neg \ \exists x \in A \ \symbol{92} \ x \% 2 \neq 0$ (No existe ningún impar en A) 
    \item $ \exists x \in A \ \symbol{92} x = 4$ (Existe un elemento en A que es exactamente 4)
\end{itemize}

\end{document} 

